% Options for packages loaded elsewhere
% Options for packages loaded elsewhere
\PassOptionsToPackage{unicode}{hyperref}
\PassOptionsToPackage{hyphens}{url}
\PassOptionsToPackage{dvipsnames,svgnames,x11names}{xcolor}
%
\documentclass[
  portuguese,
  letterpaper,
  DIV=11,
  numbers=noendperiod]{scrarticle}
\usepackage{xcolor}
\usepackage[left=3cm,top=3cm,bottom=2cm,right=2cm]{geometry}
\usepackage{amsmath,amssymb}
\setcounter{secnumdepth}{-\maxdimen} % remove section numbering
\usepackage{iftex}
\ifPDFTeX
  \usepackage[T1]{fontenc}
  \usepackage[utf8]{inputenc}
  \usepackage{textcomp} % provide euro and other symbols
\else % if luatex or xetex
  \usepackage{unicode-math} % this also loads fontspec
  \defaultfontfeatures{Scale=MatchLowercase}
  \defaultfontfeatures[\rmfamily]{Ligatures=TeX,Scale=1}
\fi
\usepackage{lmodern}
\ifPDFTeX\else
  % xetex/luatex font selection
\fi
% Use upquote if available, for straight quotes in verbatim environments
\IfFileExists{upquote.sty}{\usepackage{upquote}}{}
\IfFileExists{microtype.sty}{% use microtype if available
  \usepackage[]{microtype}
  \UseMicrotypeSet[protrusion]{basicmath} % disable protrusion for tt fonts
}{}
\makeatletter
\@ifundefined{KOMAClassName}{% if non-KOMA class
  \IfFileExists{parskip.sty}{%
    \usepackage{parskip}
  }{% else
    \setlength{\parindent}{0pt}
    \setlength{\parskip}{6pt plus 2pt minus 1pt}}
}{% if KOMA class
  \KOMAoptions{parskip=half}}
\makeatother
% Make \paragraph and \subparagraph free-standing
\makeatletter
\ifx\paragraph\undefined\else
  \let\oldparagraph\paragraph
  \renewcommand{\paragraph}{
    \@ifstar
      \xxxParagraphStar
      \xxxParagraphNoStar
  }
  \newcommand{\xxxParagraphStar}[1]{\oldparagraph*{#1}\mbox{}}
  \newcommand{\xxxParagraphNoStar}[1]{\oldparagraph{#1}\mbox{}}
\fi
\ifx\subparagraph\undefined\else
  \let\oldsubparagraph\subparagraph
  \renewcommand{\subparagraph}{
    \@ifstar
      \xxxSubParagraphStar
      \xxxSubParagraphNoStar
  }
  \newcommand{\xxxSubParagraphStar}[1]{\oldsubparagraph*{#1}\mbox{}}
  \newcommand{\xxxSubParagraphNoStar}[1]{\oldsubparagraph{#1}\mbox{}}
\fi
\makeatother


\usepackage{longtable,booktabs,array}
\usepackage{calc} % for calculating minipage widths
% Correct order of tables after \paragraph or \subparagraph
\usepackage{etoolbox}
\makeatletter
\patchcmd\longtable{\par}{\if@noskipsec\mbox{}\fi\par}{}{}
\makeatother
% Allow footnotes in longtable head/foot
\IfFileExists{footnotehyper.sty}{\usepackage{footnotehyper}}{\usepackage{footnote}}
\makesavenoteenv{longtable}
\usepackage{graphicx}
\makeatletter
\newsavebox\pandoc@box
\newcommand*\pandocbounded[1]{% scales image to fit in text height/width
  \sbox\pandoc@box{#1}%
  \Gscale@div\@tempa{\textheight}{\dimexpr\ht\pandoc@box+\dp\pandoc@box\relax}%
  \Gscale@div\@tempb{\linewidth}{\wd\pandoc@box}%
  \ifdim\@tempb\p@<\@tempa\p@\let\@tempa\@tempb\fi% select the smaller of both
  \ifdim\@tempa\p@<\p@\scalebox{\@tempa}{\usebox\pandoc@box}%
  \else\usebox{\pandoc@box}%
  \fi%
}
% Set default figure placement to htbp
\def\fps@figure{htbp}
\makeatother


% definitions for citeproc citations
\NewDocumentCommand\citeproctext{}{}
\NewDocumentCommand\citeproc{mm}{%
  \begingroup\def\citeproctext{#2}\cite{#1}\endgroup}
\makeatletter
 % allow citations to break across lines
 \let\@cite@ofmt\@firstofone
 % avoid brackets around text for \cite:
 \def\@biblabel#1{}
 \def\@cite#1#2{{#1\if@tempswa , #2\fi}}
\makeatother
\newlength{\cslhangindent}
\setlength{\cslhangindent}{1.5em}
\newlength{\csllabelwidth}
\setlength{\csllabelwidth}{3em}
\newenvironment{CSLReferences}[2] % #1 hanging-indent, #2 entry-spacing
 {\begin{list}{}{%
  \setlength{\itemindent}{0pt}
  \setlength{\leftmargin}{0pt}
  \setlength{\parsep}{0pt}
  % turn on hanging indent if param 1 is 1
  \ifodd #1
   \setlength{\leftmargin}{\cslhangindent}
   \setlength{\itemindent}{-1\cslhangindent}
  \fi
  % set entry spacing
  \setlength{\itemsep}{#2\baselineskip}}}
 {\end{list}}
\usepackage{calc}
\newcommand{\CSLBlock}[1]{\hfill\break\parbox[t]{\linewidth}{\strut\ignorespaces#1\strut}}
\newcommand{\CSLLeftMargin}[1]{\parbox[t]{\csllabelwidth}{\strut#1\strut}}
\newcommand{\CSLRightInline}[1]{\parbox[t]{\linewidth - \csllabelwidth}{\strut#1\strut}}
\newcommand{\CSLIndent}[1]{\hspace{\cslhangindent}#1}

\ifLuaTeX
\usepackage[bidi=basic]{babel}
\else
\usepackage[bidi=default]{babel}
\fi
% get rid of language-specific shorthands (see #6817):
\let\LanguageShortHands\languageshorthands
\def\languageshorthands#1{}


\setlength{\emergencystretch}{3em} % prevent overfull lines

\providecommand{\tightlist}{%
  \setlength{\itemsep}{0pt}\setlength{\parskip}{0pt}}



 


\KOMAoption{captions}{tableheading,figureheading}
\makeatletter
\@ifpackageloaded{caption}{}{\usepackage{caption}}
\AtBeginDocument{%
\ifdefined\contentsname
  \renewcommand*\contentsname{Índice}
\else
  \newcommand\contentsname{Índice}
\fi
\ifdefined\listfigurename
  \renewcommand*\listfigurename{Lista de Figuras}
\else
  \newcommand\listfigurename{Lista de Figuras}
\fi
\ifdefined\listtablename
  \renewcommand*\listtablename{Lista de Tabelas}
\else
  \newcommand\listtablename{Lista de Tabelas}
\fi
\ifdefined\figurename
  \renewcommand*\figurename{Figura}
\else
  \newcommand\figurename{Figura}
\fi
\ifdefined\tablename
  \renewcommand*\tablename{Tabela}
\else
  \newcommand\tablename{Tabela}
\fi
}
\@ifpackageloaded{float}{}{\usepackage{float}}
\floatstyle{ruled}
\@ifundefined{c@chapter}{\newfloat{codelisting}{h}{lop}}{\newfloat{codelisting}{h}{lop}[chapter]}
\floatname{codelisting}{Listagem}
\newcommand*\listoflistings{\listof{codelisting}{Lista de Listagens}}
\makeatother
\makeatletter
\makeatother
\makeatletter
\@ifpackageloaded{caption}{}{\usepackage{caption}}
\@ifpackageloaded{subcaption}{}{\usepackage{subcaption}}
\makeatother
\usepackage{bookmark}
\IfFileExists{xurl.sty}{\usepackage{xurl}}{} % add URL line breaks if available
\urlstyle{same}
\hypersetup{
  pdftitle={Lista 01 - Planejamento e Análise de Experimentos (MAE0316)},
  pdfauthor={Caio M. de Almeida - 15444560; Eduardo Yukio G. Ishihara - 15449012; Gustavo S. Garone - 15458155; Ian B. Loures - 15459667; João Victor G. de Sousa - 15463912},
  pdflang={pt},
  colorlinks=true,
  linkcolor={blue},
  filecolor={Maroon},
  citecolor={Blue},
  urlcolor={Blue},
  pdfcreator={LaTeX via pandoc}}


\title{Lista 01 - Planejamento e Análise de Experimentos (MAE0316)}
\author{Caio M. de Almeida - 15444560 \and Eduardo Yukio G. Ishihara -
15449012 \and Gustavo S. Garone - 15458155 \and Ian B. Loures -
15459667 \and João Victor G. de Sousa - 15463912}
\date{3 de setembro de 2025}
\begin{document}
\maketitle


\vspace{-0.5cm}\noindent\rule{\textwidth}{1pt}

Nesta lista, usaremos ``\(.\)'' como separador decimal.

\section{Exercício 1}\label{exercuxedcio-1}

\subsection{Item a}\label{item-a}

Analisamos um estudo de coorte sobre os efeitos da ``COVID longa'' ou
Síndrome pós-COVID-19 no trabalho de ROCHA et al. (2024). O estudo
utilizou o método de coorte ambidirecional com indivíduos de três
hosítais de Cuiabá. Observações foram colhidas do prontuário desses
pacientes e posteriormente \(6\) e \(12\) meses após alta hospitalar por
telefone. Foram perguntados fatores socioeconômicos dos indivíduos, além
de sintomas comuns da ``COVID longa'', como fadiga e problemas de
memória.

Na análise de dados, foram consideradas pelos pesquisadores comorbidades
como hipertensão, diabete, obesidade e doenças cardíacas. Para os
sintomas, classificaram como musculares, neuropsiquiátricos,
dermatológicos, cardiovasculares e pulmonares.

Acreditamos que o estudo tenha uma base metodológica, no geral, sólida,
mas que a dependência na descrição por telefone dos entrevistados pode
ter comprometido a integridade das conclusões. Isso foi parcialmente
reconhecido pelos autores, que perceberam que a obesidade era
subrepresentada quando comparada com o IMC calculado a partir da altura
e peso dos entrevistados. Isto é, dos entrevistados, quando perguntavam
se estavam com sobrepeso, responderam positivamente apenas \(11\) dos
\(46\) identificados com obesidade pelo IMC. Os autores consideraram
isso na análise dos dados, mas não há discussão sobre outras
imprecisões.

\subsection{Item b}\label{item-b}

Pode ser que haja diferença entre as conclusões dos estudos. No primeiro
estudo, fatores como viés de seleção (por exemplo, selecionar pacientes
que optaram pelo estudo, pois utilizam a medicina convencional, e
paciente que optaram por não usar o remédio, pois utilizam
exclusivamente a medicina alternativa) e menor aleatorização. No segundo
estudo, ao selecionar previamente o grupo, e então efetuar um ensaio
randomizado, espera-se que o processo de amostragem aleatória faça com
que os dois grupos, na média, tenham indivíduos semelhantes. Logo, a
amostragem aleatória forneceria resultados mais robustos e conclusões
mais acuradas ao atenuar fatores intrínsecos das unidades amostrais
dentre os grupos, como a gravidade da doença, estilos de vida ou fatores
biológicos.

\subsection{Item c}\label{item-c}

Um resultado possível, que não considera vieses dos investigadores ou
resultados, é que, no primeiro estudo, o remédio já estava sendo
aplicado em pacientes nos estágios avançados da doença como medida
emergencial, enquanto no segundo estudo o remédio foi aplicado tanto em
pacientes com doença avançada quanto leve ou moderada. Logo, uma taxa de
cura menor no primeiro estudo poderia ser explicada pela ineficiência do
medicamento de conter a doença no estágio avançado, enquanto funciona
bem no geral, como aponta o segundo estudo.

\section{Exercício 2}\label{exercuxedcio-2}

É apresentado um estudo prospectivo, aleatorizado, experimental
(analítico) e controlado por grupo controle, sobre a aplicação de
determinado tratamento - oxigenação específica - em ratos com diabete
(população objetivo). Como hipótese, buscaram discobrir se existe efeito
do oxigênio hiperbárico na cicratização de feridas cirúrgicas. Como
fator extrínsico explicativo consta a exposição ou não - caracterizando
dois níveis para esse fator - das unidades experimentais (ratos com
diabete induzida) ao tratamento com oxigênio hiperbárico, uma exposição
com intervenção. Um possível fator intrínseco é a própria variação
biológica dos ratos, que pode acelerar ou retardar a cicatrização. A
repetição foi realizada com diversos ratos por grupo (que receberam ou
não o tratamento). Variações externas como níveis glicêmicos inadequados
no sangue foram tratadas ao desconsiderar \(6\) desses ratos. Variações
acidentais tentaram ser minimzadas por preocauções dos pesquisadores.
Por se tratar de um estudo experimental, não podemos descrever unidades
observacionais, nem se encaixa como longitudinal ou transversal.

\section{Exercício 3}\label{exercuxedcio-3}

\subsection{Item a}\label{item-a-1}

O contraste \(p=1\) pode ser usado para testar se o efeito da dose zero
difere do efeito da dose um, enquanto o contraste \(p=2\) pode ser usado
para comparar o efieto da dose dois conjuntamente com as doses um e
dois. Ambos os constrastes seriam nesse exemplo usados para encontrar a
dose limiar de efeito.

\subsection{Item b}\label{item-b-1}

Assumindo tamanho amostral igual para as doses, os constrastes são
ortogonais:

\[
\begin{aligned}
(-1) \cdot (-1) + 1 \cdot (-1) + 0 \cdot 2 &= 0 \\
(-1) \cdot (-1) + 1 \cdot (-1) + 0 \cdot (-1) + 0 \cdot 3 &= 0 \\
(-1) \cdot (-1) + (-1) \cdot (-1) + 2 \cdot (-1) + 0 \cdot 3 &= 0
\end{aligned}
\]

\subsection{Item c}\label{item-c-1}

A hipótese \(\mathcal{H}_2\) é a mais apropriada para testar a dose
limiar uma vez que busca a primeira dose \(j\) a possuir uma média
significativamente maior que as anteriores, que são iguais, assumindo
doses crescentes. Caso não exista, caímos na hipótese de não efeito
\(\mathcal{H}_0\). Não sugeririamos outra hiótese para este problema.

\subsection{Item d}\label{item-d}

\section{Exercício 4}\label{exercuxedcio-4}

\section{Referências}\label{referuxeancias}

\phantomsection\label{refs}
\begin{CSLReferences}{0}{1}
\bibitem[\citeproctext]{ref-rocha_sindrome_2024}
ROCHA, R. P. S. et al.
\href{https://doi.org/10.1590/0102-311XPT027423}{Síndrome pós-{COVID}-19
entre hospitalizados por {COVID}-19: estudo de coorte após 6 e 12 meses
da alta hospitalar}. \textbf{Cadernos de Saúde Pública}, v. 40, p.
e00027423, 2024.

\end{CSLReferences}




\end{document}
